\chapter{Planification du projet}

\section{Planning du projet}

Nous avons \'etabli le planning de la figure \ref{fig:planning} pour notre projet. Les barres gris\'ees correspondent \`a la plannification initiale, les barres bleues au 
planning r\'eel.
\newline
Notre travail s'est effectu\'e en trois \'etapes. Premi\`erement, l'analyse du besoin et la r\'edaction du cahier des charges fonctionnel.
Deuxi\`emement, la phase de conception, de r\'ealisation et de test de notre solution, estim\'ee \`a 64 jours.
Enfin, la phase de conception du rapport et du poster.
Comme notre biblioth\`eque est constitu\'ee de dix utilitaires, nous les avons r\'eparti entre nous trois. Ainsi, nous avons chacun effectu\'e les diff\'erentes
\'etapes de la phase de d\'eveloppement. C'est pourquoi, aucun nom n'est associ\'e \`a une t\^ache en particulier.

\section{Analyse des \'ecarts}

La phase de codage a dur\'e deux mois au lieu de six semaines. Cela s'explique par les nombreuses am\'eliorations apport\'ees au code au fur et \`a mesure.
Nous avons rattrap\'e ce retard en effectuant la cr\'eation des fichiers binaires en une semaine au lieu de deux semaines.	 
Nous avons commenc\'e la r\'edaction du rapport plus tard afin de finir le plus rapidement possible la phase de d\'eveloppement.
Nous avions \'egalement pr\'evu deux semaines pour cr\'eer le poster mais nous avons pu le finir en une semaine.
\newline
Dans la partie << Suivi du projet >>, vous pouvez constater que nous n'avons pas pu effectuer toute les r\'eunions pr\'evues. A cause de l'indisponibilit\'e de notre
tuteur lors de conf\'erences, nous avons d\^u d\'ecaler nos r\'eunions. Toutefois, nous avons pu communiquer facilement \`a distance.
On constate donc que nous avons globalement suivi la plannification initiale.

\newpage
\begin{figure}[h]
\begin{minipage}[c]{0mm}
\includegraphics[width=15cm,height=21cm]{planning.eps}
\end{minipage}
\caption{Planning du projet}
\label{fig:planning}
\end{figure}

