\chapter*{Conclusion}

Au cours de notre projet, nous avons appris \`a suivre une d\'emarche de conception sp\'ecifique en vue d'aboutir \`a la cr\'eation
puis \`a la publication d'un paquet Debian. Notre travail se basait sur une biblioth\`eque de calcul num\'erique existante et d\'ej\`a
distribu\'ee num-utils. Nous avons pu montrer notre capacit\'e \`a analyser un code \'ecrit par un autre d\'eveloppeur, puis \`a l'optimiser dans 
un autre langage, en C, pr\'esentant des avantages mais \'egalement des inconv\'enients. Nous avons appris, progressivement, \`a utiliser des outils
de d\'eveloppement : le gestionnaire de version, la cr\'eation de pages de manuel (man pages), les autotools puis les outils de cr\'eation de paquets Debian. Finalement, nous obtenons une premi\`ere
version de la biblioth\`eque num-utils-ng.
\newline
\newline
Maintenant, nous devons continuer \`a am\'eliorer notre biblioth\`eque num\'erique. Il nous faut ajouter de nouvelles fonctionnalit\'es utiles et \'egalement 
corriger les bugs qui subsistent encore dans la premi\`ere version de notre biblioth\`eque num\'erique. Il nous faut aussi optimiser certains des 
algorithmes impl\'ement\'es, qui quelquefois sont un plus lents ou utilisent plus de m\'emoire que la version d\'ej\`a publi\'ee. Enfin il nous restera \`a
diffuser notre paquet. Nous avons cr\'e\'e un site internet (en construction actuellement) pour nous permettre de proposer notre distribution open source,
d'offrir une assistance technique aux futurs utilisateurs et de r\'epondre \`a leurs suggestions
\newline
\newline
L'objectif final \'etant ainsi la publication, dans les vraies archives Debian, du paquet num-utils-ng.
