\chapter{Introduction}
\label{chap:introduction}

\section{Motivations }

Le projet 11\citep{projet11}, intitul\'e << Optimisation de biblioth\`eque de calcul num\'erique >>, est un projet informatique bas\'e sur le langage C qui vise \`a am\'eliorer 
les performances de calcul de la biblioth\`eque num-utils\citep{numutils}, travaillant sur des flots de donn\'es\citep{dataflux}.
Il aboutira au d\'eveloppement d'un vrai logiciel qui sera publi\'e dans les archives Debian\citep{debian}.

Ceci est \'evidemment une source de motivation non n\'egligeable. Mais \`a travers cet aspect concret de notre projet, c'est tout l'environnement d'un 
projet informatique , les contraintes impos\'es , l'interface existante entre les d\'eveloppeurs, le travail d'optimisation, qui sont les raisons 
d'\^etre de ce travail.
Il s'inscrit alors dans un contexte pr\'ecis : le d\'eveloppement d'un projet informatique en vue de sa publication.

\section{Objectifs}

Bien que le projet repose sur une optimisation de performances, l'objectif principal reste la mise en place des outils n\'ecessaires en vue de la 
r\'ealisation et de la publication d'un logiciel dans les archives Debian.  

En effet derri\`ere la probl\'ematique d'optimisation se cache une autre probl\'ematique qui sera le point de d\'epart de notre projet : comment mettre
 en oeuvre un projet informatique ? 
Notre d\'emarche consistera en premier lieu en la d\'ecouverte et l'utilisation des outils n\'ecessaire \`a la r\'ealisation d'un projet informatique afin 
de mettre en ligne une biblioth\`eque num-utils cod\'e en C.
\newline 	
Tout d'abord une impl\'ementation des algorithmes sera fournit en langage C avec l'ensemble des outils de d\'eveloppement mis en oeuvre. Ensuite, 
un paquet Debian fournissant une distribution open source sera mise en ligne. 
Outre les d\'ecouvertes techniques, cette premi\`ere approche sera l'occasion de constater toute l'importance de la forme ( page de manuel, readme,...) 
lors de la conception d'un logiciel informatique publi\'e.

La seconde \'etape du projet sera l'optimisation des performances de calculs de nos codes C . Cette phase, plus courte, se veut diff\'erente de la 
premi\`ere : elle n\'ecessite avant tout un travail de r\'eflexion. C'est  un travail de fond o\`u l'objectif principal sera une division par 10 du temps
 d'ex\'ecution. 

Notre d\'emarche sera donc constitu\'ee de deux \'etapes : la mise en place d'un socle de travail, puis le travail de fond d'optimisation. Ces deux phases 
s'inscrivent dans le contexte du projet qui est le d\'eveloppement d'un projet informatique afin de r\'epondre \`a la probl\'ematique explicite du sujet 
qui est l'optimisation d'une biblioth\`eque de calcul num\'erique.


