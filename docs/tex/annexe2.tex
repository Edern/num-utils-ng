\chapter{Analyse personnelle du projet}

\begin{itemize}
\item [\textbullet] \Large \textbf{Reuven Benichou}
\newline
\normalsize
\item [\textbullet] \Large \textbf{Edern Hotte}
\newline
\normalsize
J'ai trouv\'e ce projet tr\`es int\'eressant. Je dois avouer qu'au d\'epart, j'avais des appr\'ehensions quant \`a la quantit\'e de code que nous avions \`a livrer qui s'est finalement 
r\'ev\'el\'e ne pas \^etre si \'enorme que \c ca. J'ai appris beaucoup de choses : les \'etapes de la publication d'un paquet Debian, l'\'elaboration et l'\'ecriture de scripts de tests 
en bash et comment s'assurer qu'un programme s'ex\'ecutera comme pr\'evu m\^eme si l'utilisateur ne l'utilise pas de fa\c con normale. J'ai d\'ecouvert des outils tr\`es utiles au d\'eveloppement, par exemple Valgrind qui permet de retrouver
d'o\`u proviennent certaines erreurs de segmentations et de contr\^oler les fuites de m\'emoire. J'ai aussi appris \`a utiliser les Makefiles permettent un tr\`es grand gain
de temps gr\^ace \`a l'automatisation de certaines taches comme la compilation des sources ou la suppression des fichiers interm\'ediaires. Certaines fonctions du C m'\'etaient m\'econnues avant ce projet,
getopt, qsort ou encore perror sont de nouveaux mots dans mon vocabulaire. Notre encadrant nous a bien aid\'e en nous suivant et nous conseillant tout au long du d\'eveloppement et l'ambiance du groupe \'etait plut\^ot bonne.\newline
\item [\textbullet] \Large \textbf{Flavien Moullec}
\newline
\normalsize
Ce projet a \'et\'e pour moi int\'eressant et instructif. Au d\'ebut, je ne savais pas exactement quel type de travail nous aurions \`a effectuer.
Comme notre travail \'etait de suivre les \'etapes amenant \`a la cr\'eation d'un paquet Debian, je ne connaissais pas les outils de
d\'eveloppement que nous avons utilis\'e. Ce projet m'a permis de cr\'eer le planning du projet et de me rendre compte de la difficult\'e d'en r\'ealiser un
et de le respecter ensuite. Je suis assez content d'avoir appris \`a utiliser des outils de d\'eveloppement comme les autotools ou le programme de cr\'eation de 
paquet Debian, que j'ai trouv\'e assez int\'eressant.

J'ai trouv\'e certaines parties du projet, notamment les script de tests assez difficiles \`a comprendre, ou dans l'utilisation de commandes Unix qui pouvaient
simplifier notre code quelquefois, si nous les avions connues au d\'ebut.

Notre encadrant nous a bien suivi, en nous donnant des conseils techniques et disponible pour r\'epondre \`a nos questions.
Je suis content de mon groupe car nous \'etions dans l'ensemble plut\^ot motiv\'e pour faire avancer le projet.


\end{itemize}
