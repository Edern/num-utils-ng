\chapter{D\'emarche de conception effectu\'ee}
\label{chap:outils utilises}

\section{Utilisation du gestionnaire de version }
\subsection{Pr\'esentation}

La r\'ealisation de notre projet informatique a en premier lieu n\'ecessit\'e la mise en place d'une interface centralisant tous nos
documents en temps r\'eel. Pour cela nous avons utilis\'e ce que l'on appelle un gestionnaire de version. Plusieurs logiciels de gestions 
de versions existent aujourd'hui : CVS, Subversion, Mercurial, Bazar et enfin Git, ayant chacun leurs avantages et inconv\'enients. 

Ils permettent d'archiver et de conserver les diff\'erentes \'etapes de d\'eveloppement d'un projet. Ainsi il est possible de pouvoir revenir
\`a une version ant\'erieure \`a tout moment. Ils permettent \'egalement de visualiser les diff\'erences entre les r\'evisions. Cela permet un  
travail collaboratif tr\`es efficace : chaque d\'eveloppeur dispose du projet en local et peut quand il le souhaite les partager sur un serveur.

Nous avons utilis\'e pour notre projet, le logiciel Git.

\subsection{Le logiciel Git}
\index{git@Git}
Git est un logiciel libre cr\'ee par Linus Torvalds, le fondateur de Linux. C'est un logiciel de version distribu\'e c'est \`a dire qu'il n'est
 pas n\'ecessaire d'utiliser un serveur pour partager nos documents : on peut chacun se synchroniser entre nous.

Au premier abord l'utilisation de git n'est pas \'evidente : toutes les manipulations se font \`a travers la console et le vocabulaire utilis\'e 
est totalement nouveau. 

Apr\`es l'installation de git, il est n\'ecessaire de s'enregistrer (mail et nom) et de cr\'eer un dossier de travail dans lequel nous 
initialisons git en utilisant la commande
\begin{verbatim} git init \end{verbatim}
Ce dossier de travail sera le d\'ep\^ot de notre projet informatique. L'ajout d'un fichier dans le dossier reconnu par git se r\'ealise
 \`a travers la commande
\begin{verbatim} git add <nom du fichier> \end{verbatim}
La commande
\begin{verbatim} git commit <nom du fichier> \end{verbatim} permet d'enregistrer la modification effectu\'ee dans le dossier de travail.
Ces 2 commandes furent utilis\'ees maintes fois lors de notre projet. 

Cependant tout cela reste local : il faudra utiliser la commande
\begin{verbatim} git push \end{verbatim}
pour rajouter au dossier centrale, contenant tous le travail des diff\'erents membres du projet, nos nouveaux fichiers. 
De m\^eme pour t\'el\'echarger ce qui fut r\'ealis\'e par les autres membres 
du projet nous avons utilis\'es la commande
\begin{verbatim} git pull \end{verbatim}

Ceci constituent les commandes de bases utilis\'es lors de notre projet. Une fois le principe assimil\'e git devient tr\`es simple d'utilisation,
 tr\`es rapide mais surtout tr\`es efficace.

Enfin une des particularit\'es de Git est l'existence de sites web collaboratifs bas\'es sur Git comme  Github ou Gitorious. 

\subsection{Github}

Github peut \^etre consid\'er\'e comme un r\'eseau social pour les programmeurs. En effet c'est un << entrep\^ot >> de projet utilisant git comme 
gestionnaire de version. Ainsi Github permet \`a un quelconque utilisateur d'intervenir dans un projet , d'utiliser ces codes sources etc.
GitHub cr\'ee une page de profil simple o\`u appara\^it notre nom, email, etc. Cette page affiche \'egalement notre activit\'e (commit, ajout de suivi,...)
 et nos d\'ep\^ots. De m\^eme nous pouvons recevoir si on le souhaite les mise \`a  jour ( commit, comment) li\'es \`a un projet quelconque.

Les pages d'un projet commencent par une s\'erie d'onglets permettant de parcourir les sources (page par d\'efaut), d'acc\'eder \`a la liste des
 commits, le network, les demandes de pull, les probl\`emes et les wikis.
La page source, qui est la plus importante, contient toute l'arborescence de notre projet, l'adresse SSH et HTTP et permet une vue simple
 de l'avancement de notre travail.
Les autres pages furent moins utilis\'ees car secondaire. 

Cette collaboration entre Git et Github a \'et\'e essentielle pour notre projet et a permis un suivi clair et pr\'ecis de son avancement.
\newline
\newline
\newline
\newline
ajouter dans la biblio :
\newline
https://github.com/

http://www.crunchbase.com/company/github

http://fr.wikipedia.org/wiki/GitHub

http://www.siteduzero.com/tutoriel-3-254198-gerez-vos-codes-source-avec-git.html

http://www.unixgarden.com/index.php/administration-systeme/git-it


\section{Automatisation de la compilation de notre projet}
\subsection{Compilation avec des Makefiles}

La compilation d'un fichier s'effectue en langage C s'effectue de la mani\`ere suivante :
\begin{verbatim} gcc -c numaverage.c\end{verbatim}
\begin{verbatim} gcc -o numaverage numaverage.o\end{verbatim}

La premi\`ere \'etape correspondant \`a la transformation en code binaire, la seconde \`a l'\'edition des liens.

Pour chaque ex\'ecutable, \`a chaque modification du code source, il faut \'ecrire ces quelques lignes dans un terminal.
Nous avons donc utilis\'e le Makefile pour compiler notre code (voir \ref{fig:exemple_makefile}), afin d'automatiser cett \'etape.
En effect, la commande << \textbf{make} >> cr\'ee l'ex\'ecutable pour tous les fichiers d\'efinis dans le makefile. La commande
<< \textbf{make clean} >> supprimer tous les fichiers temporaires.
\newline
\begin{figure}[h] 
\begin{center}

\begin{minipage}[|c|]{0.7\linewidth}
\begin{verbatim}
UTILS = numaverage numbound
CC = gcc CFlAGS = -WALL
LIBS = -lm
all : $(UTILS).o
  for util in $(UTILS) ; do \
    $$(CC) $(CFLAGS) -o $$util $$util.o ; done

$(UTILS).o:
  for util in $(UTILS) ; do \
    $$(CC) $(CFLAGS) -c $$util.c ; done

clean :
  $(RM) $(EXEC).o

\end{verbatim}
\end{minipage}
\end{center}
\caption{Syntaxe d'un de nos premiers Makefile}
\label{fig:exemple_makefile}
\end{figure}

Toutefois, l'\'ecriture manuelle d'un Makefile devient impossible si l'on veut ult\'erieurement cr\'eer un paquet debian.
En effet, le principal probl\`eme de ce type de Makefile est qu'il est identique pour chaque syst\`eme d'exploitation. Or certains fichiers
syst\`emes sont n\'ecessaires pour l'utilisation de notre biblioth\`equ num-utils-ng, mais ils ne sont pas forc\'ement dans le r\'epertoire pour
chaqu syst\`eme d'exploitation.
\newline 	
Par cons\'equent, il nous a \'et\'e propos\'e d'utiliser des \textbf{autotools} pour cr\'eer des Makefiles << dynamiques >>, qui d\'ependent du syst\`eme
d'exploitation de l'utlisateur.

\subsection{G\'en\'eration automatique de Makefiles : utilisation des autotools}

L'utilisation d'autotools n\'ecessite de suivre une d\'emarche sp\'ecifique afin de cr\'eer nos Makefiles. Nous avons utilis\'e pour
cela trois outils : automake, autoconf et aclocal.

\section{Cr\'eation du paquet debian}