
\begin{description}
\item[D�finition 1 : Le Makefile]
est un fichier utilis� par la commande make des distributions GNU/Linux permettant l'automatisation de t�ches simples telles que la compilation.
\item[D�finition 2 : Une Distribution] 
de GNU/Linux est une version d'un syst�me d'exploitation, il y en a plusieurs car il y a plusieurs syst�mes d'exploitation utilisant le m�me noyau : Ubuntu, Debian, Red Hat.
\item[D�finition 3 : Un Paquet] 
Debian est un paquet rassemblant les sources d'un programme et comment les install�s, ainsi que les d�pendances nescessaires. On les trouve typiquement dans les archives Debian et elle serven � installer des programmes facilement sous Ubuntu ou Debian.
\item[D�finition 4 : Un Flux ou Flot de donn�es] 
est la succession de donn�es d'un certain type, elle peut se poursuivre pendant un temps ind�fini.
\item[D�finition 5 : Une Allocation] 
est l'op�ration qui consiste � r�server de la place dans la m�moire, c'est une op�ration qui demande un peu de ressoources et qui est donc � utiliser le moins possible.
\item[D�finition 6 : Le mode]
d'une s�rie de nombre est le nombre qui apparait le plus de fois dans la s�rie.
\end{description}
