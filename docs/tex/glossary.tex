\chapter*{Glossaire}
\begin{description}

\item[Allocation] : l'alloction est l'op�ration qui consiste � r�server de la place dans la m�moire, c'est une op�ration qui demande un peu de ressources et qui est donc � utiliser le moins possible.
\item[autotools] : outils fournis par le projet GNU pour faciliter la fabrication de paquets � partir du code source d'un programme.
\item[Distribution] : Une distribution de GNU/Linux est une version d'un syst�me d'exploitation, il y en a plusieurs car il y a plusieurs syst�mes d'exploitation utilisant le m�me noyau : Ubuntu, Debian, Red Hat.
\item[Flux ou Flot de donn�es] : le flot de donn�es est la succession de donn�es d'un certain type, qui peuvent se poursuivre pendant un temps ind�fini.
\item [fuite de m�moire] : occupation croissante et non contr�l�e ou non d�sir�e de la m�moire d'un ordinateur. 
\item[Gestionnaire de version] : logiciel qui permet de stocker un ensemble de fichiers en conservant la chronologie de toutes les modifications qui ont �t� effectu�es dessus.
\item[Langage compil�] : langage dans lequel le code est traduit en langage binaire pour une compr�hension direct du processeur. Un programme �crit en langage compil� est traduit en instructions lisibles par la machine et peut �tre ex�cut� ind�pendamment de tout autre programme.
\item[Langage interpr�t�] : langage dans lequel le code n'est pas compil� . Un programme �crit en langage interpr�t� n'est pas ex�cut� directement par la machine mais par un autre programme.
\item[logiciel open source] : logiciel dont l'utilisation, l'�tude, la modification et la duplication en vue de sa diffusion sont permises, techniquement et l�galement.
\item[Makefile] : un Makefile est un fichier utilis� par la commande make des distributions GNU/Linux permettant l'automatisation de t�ches simples telles que la compilation.
\item[mode] : le mode d'une s�rie de nombres est le nombre qui apparait le plus de fois dans la s�rie.
\item[Paquet] : un paquet Debian est un paquet rassemblant les sources d'un programme et comment les installer, ainsi que les d�pendances n�cessaires. On les trouve typiquement dans les archives Debian et elle servent � installer des programmes facilement sous Ubuntu ou Debian.

\end{description}
