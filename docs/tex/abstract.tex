\section*{R\'esum\'e}

La gestion des flots de donn\'ees est tr\`es importante dans beaucoup de domaines de d\'eveloppement informatique, un d\'eveloppeur peut avoir besoin 
d'effectuer des calculs simples sur ceux-ci, il a donc besoin d'outils lui permettant d'acc\'eder aux valeurs qu'il recherche. La biblioth\`eque num-utils
 cr\'e\'ee par Suso Banderas en 2002 propose dix utilitaires permettant d'effectuer ces calculs. Toutefois, \`a cause du langage interpr\'et\'e utilis\'e, 
le Perl, ses performances globales ne sont pas toujours satisfaisantes. Le but de ce projet \'etait de proposer une alternative \`a cette biblioth\`eque, 
\'ecrite en C pour avoir des choix de conceptions diff\'erents et afin d'\^etre plus rapide. Dans ce rapport technique nous pr\'esenterons dans un premier temps 
la biblioth\`eque compos\'ee des dix utilitaires que nous avons impl\'ement\'e en respectant l'esprit de l'ancienne biblioth\`eque ; nous avons fait en sorte que
 l'utilisation de la nouvelle biblioth\`eque se fasse de la m\^eme mani\`ere que l'ancienne afin de proposer une vraie alternative sous la forme d'un paquet
 Debian aux anciens utilisateurs. Dans un deuxi\`eme temps nous pr\'esenterons les nombreux outils que nous avons \'et\'e amen\'e \'a utiliser pendant le 
d\'eveloppement : du simple gestionnaire de version aux outils de tests en passant par les utilitaires facilitant le d\'eveloppement en automatisant certaines t\^aches.
 Enfin nous finirons par la comparaison des performances de l'ancienne biblioth\`eque et de la nouvelle en analysant les principales diff\'erences entre celles-ci et en expliquant nos choix de conception.

\section*{Abstract}

The management of data streams is very important in many fields of computer development, a developer may need to perform simple calculations on them, he needs tools
 to access the values he wants. The num-utils library created in 2002 by Suso Banderas offers ten utilities to perform these calculations but because of the
 interpreted language used, Perl, its efficiency is not as good as it could be. The aim of this project was to offer an alternative to this library, written in
 C and with different design choices in order to improve its speed. In this report we will first present  the library that includes the ten utilities we implemented,
 respecting the spirit of the old library we try to ensure that the use of the new library is the same as the older one in order to offer a real alternative to former
 users. Then we will introduce the tools we have been brought to use during the development : from the version manager to the tools we used to make the tests and the
 ones to facilitate development by automating certain tasks. Eventually we will finish by comparing the efficiency of the old library and the new one by analyzing the
 differences between them and by explaining our design choices.

\section*{Mot-cl\'es}

GNU, C, num-utils, autotools, debian
