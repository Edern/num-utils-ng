\chapter{La biblioth\`eque num-utils}
\label{chap:La bibliotheque num-utils}

\section{Pr\'esentation}
\index{Perl}
\index{num-utils}
La biblioth\`eque de calcul num\'erique qu'il nous fallait am\'eliorer s'appelle num-utils. Elle est \'ecrite dans un langage
interpr\'et\'e, le Perl.
Le paquet num-utils est inclus dans la distribution officielle de Debian.

Elle impl\'emente un certain nombre d'algorithmes num\'eriques travaillant sur des flots de donn\'ees.
Ces algorithmes sont r\'epartis dans dix programmes diff\'erents s'intitulant : numaverage, numbound, numgrep, numinterval, numnormalize,
numprocess, numrandom, numrange, numround et numsum.

La premi\`ere am\'elioration que l'on pouvait apporter \`a la biblioth\`eque \'etait de passer d'un langage interpr\'et\'e en un langage compil\'e.
Nous avons ainsi cod\'e, en langage C, l'ensemble de ces utilitaires.

Nous avons respect\'e dans la mesure du possible les m\^emes notations et les m\^emes noms d'options que ceux de la biblioth\`eque num-utils.
Notre nouvelle biblioth\`eque s'appelle \textbf{num-utils-ng}.

\section{Fonctionnalit\'es propos\'ees par ces dix programmes}

\subsection{Points communs de chaque utilitaire}
Cette partie explique en d\'etail les fonctionnalit\'es propos\'es par ces diff\'erents programmes.
Les dix programmes constituant la biblioth\`eque num-utils-ng s'ex\'ecutent en ligne de commande dans un terminal Unix (shell).
Pour chacun de ces programmes, except\'e numrandom qui ne prend aucune donn\'ee en entr\'ee, les donn\'ees sont lues \`a partir 
de l'entr\'ee standard (stdin) ou \`a partir d'un fichier.
Ces dix programmes proposent chacun une option << \textbf{-h} >> permettant \`a l'utilisateur d'obtenir une aide sur le fonctionnement du 
programme sous la forme d'une page de manuel (man pages en anglais).

\subsection{Calculs simples sur des nombres d\'ecimaux}

Cette partie regroupe trois programmes : numaverage, numnormalize et numsum.
\newline
\begin{itemize}\index{numaverage}
 \item[\textbullet] \textbf{numaverage :} numaverage retourne la moyenne des nombres relatifs pass\'es en entr\'ee.
Ce programme propose cinq options :
\begin{itemize}
  \item[-] << \textbf{-i} >> : retourne la partie enti\`ere de la moyenne,
  \item[-] << \textbf{-I} >> : retourne la partie d\'ecimale de la moyenne,
  \item[-] << \textbf{-m} >> : retourne le nombre qui appara\^it le plus souvent,
  \item[-] << \textbf{-M} >> : retourne le nombre m\'edian,
  \item[-] << \textbf{-l} >> : retourne le nombre m\'edian le plus petit (utile quand le nombre total d\'el\'ements est pair).
\newline
\end{itemize}\index{numnormalize}
\item[\textbullet] \textbf{numnormalize :} numnormalize prend un ensemble de nombres en entr\'ee et retourne l'ensemble de ces nombres, 
normalis\'es entre 0 et 1 par d\'efaut. Il propose une seule option :
\begin{itemize}
 \item[-] << \textbf{-R <n1>..<n2>} >> : la normalisation est effectu\'ee entre les nombres <n1> et <n2>.
\newline
\end{itemize}\index{numsum}
 \item[\textbullet] \textbf{numsum :} numsum retourne la somme des nombres relatifs pass\'es en entr\'ee.
Six options sont propos\'ees :
\begin{itemize}
 \item[-] << \textbf{-i} >> : retourne la partie enti\`ere de la somme,
 \item[-] << \textbf{-I} >> : retourne la partie d\'ecimale de la somme,
 \item[-] << \textbf{-c} >> : retourne la somme de chaque colomne,
 \item[-] << \textbf{-r} >> : retourne la somme de chaque ligne,
 \item[-] << \textbf{-x <n>} >> : retourne la somme des colomnes d\'efinies dans l'ensemble <n>,
 \item[-] << \textbf{-y <n>} >> : retourne la somme des lignes d\'efinies dans l'ensemble <n>.
\end{itemize}
\end{itemize}

\subsection{Comparaisons de nombres d\'ecimaux}

Cette partie regroupe trois programmes : numbound, numinterval et numround.
\newline
\begin{itemize}\index{numbound}
 \item[\textbullet]  \textbf{numbound :} numbound renvoie le plus grand nombre pass\'e en entr\'ee.
Ce programme propose une seule option :
\begin{itemize}
  \item << \textbf{-l} >> : retourne le plus petit nombre pass\'e en entr\'ee.
\newline
\end{itemize}\index{numinterval}
 \item[\textbullet] \textbf{numinterval :} numinterval calcule et affiche l'intervalle entre un nombre et le suivant du flux d'entr\'ee.
Ce programme ne poss\`ede pas d'options sp\'ecifiques.
\newline\index{numround}
 \item[\textbullet] \textbf{numround :} numround arrondit le nombre d\'ecimal pass\'e en entr\'e en l'entier le plus proche.
Ce programme propose trois options :
\begin{itemize}
 \item[-] << \textbf{-c} >> : renvoie la partie enti`ere du nombre en entr\'ee,
 \item[-] << \textbf{-f} >> : renvoie l'entier imm\'ediatement sup\'erieur \`a la valeur du nombre en entr\'ee, 
 \item[-] << \textbf{-n <n>} >> : renvoie le multiple de l'entier <n> le plus proche du nombre en entr\'ee.
\end{itemize}
\end{itemize}

\subsection{G\'en\'eration et modifications de nombre d\'ecimaux}

Cette partie regroupe quatre programmes : numgrep, numprocess, numrandom et numrange.
\newline
\begin{itemize}\index{numgrep}
 \item[\textbullet] \textbf{numgrep :} numgrep retourne l'ensemble des nombres v\'erifiant la condition pass\'ee en argument. La condition est de la
forme << /expression/ >>. Les diff\'erentes expressions possibles sont les suivantes :
  \begin{itemize}
  \item[-] << \textbf{/<n1>..<n2>/} >> : retourne l'ensemble des nombres appartenant \`a l'intervalle [n1,n2], n1 et n2 \'etant des entiers.
  \item[-] << \textbf{/m<n>/} >> : retourne l'ensemble des multiples du nombre <n>,
  \item[-] << \textbf{/f<n>/} >> : retourne l'ensemble des diviseurs du nombre <n>.
  \end{itemize}
Remarque : Ces expressions peuvent \^etre combin\'ees en les s\'eparant par une virgule << \textbf{,} >>.
\newline\index{numprocess}
 \item[\textbullet] \textbf{numprocess :} numprocess effectue une liste d'op\'erations sur l'ensemble des nombres pass\'es en entr\'ee.
Les op\'erations possibles sont d\'ecrite dans le tableau 2.1 ci-dessous. Deux constantes sont \'egalement d\'efinies : << \textbf{pi} >> 
et << \textbf{e} >>, repr\'esentant respectivement $\Pi$ et $e^1$.
\begin{table}[h]
\begin{center}

\begin{tabular}{|l|c|r|}
  \hline
  Symbole &  Op\'eration\\
  \hline
  << \textbf{+} >> & Addition \\
  << \textbf{-} >> & Soustraction \\
  << \textbf{*} >> & Multiplication \\
  << \textbf{\%} >> & Division \\
  << \textbf{\^} >> & Puissance \\
  << \textbf{sqrt} >> & Racine carr\'ee \\
  << \textbf{sin} >> & Sinus \\
  << \textbf{cos} >> & Cosinus \\
  \hline
\end{tabular}
\caption{Symboles utilis\'es dans numprocess}
\end{center}
\label{tab:symboles}
\end{table}
\index{numrandom}
 \item[\textbullet] \textbf{numrandom :} numrandom retourne un nombre g\'en\'er\'e al\'eatoirement. Par d\'efaut, ce nombre est un entier entre 1 et 100.
Le programme peut prendre un argument de la forme << /expression/ >>. Les diff\'erentes expressions possibles sont les suivantes :
\begin{itemize}
 \item[-] << \textbf{/<n1>:<n2>/} >> : retourne un entier  g\'en\'er\'e al\'eatoirement entre les nombres <n1> et <n2>, avec un pas de 1,
 \item[-] << \textbf{/<n1>:<n2>i<p>/} >> : retourne un entier  g\'en\'er\'e al\'eatoirement entre <n1> et <n2>, avec un pas <p>.
\end{itemize}
Remarque : Ces expressions peuvent \^etre combin\'ees en les s\'eparant par une virgule << \textbf{,} >>.
\newline\index{numrange}
 \item[\textbullet] \textbf{numrange :} numrange affiche la liste des nombre v\'erifiant la condition pass\'ee en argument de la forme << /expression/ >>.
Les diff\'erentes expressions possibles sont les suivantes :
\begin{itemize}
 \item[-] << \textbf{/<n1>:<n2>/} >> : retourne la liste des nombres compris entre <n1> et <n2>, avec un pas de 1,
 \item[-] << \textbf{/<n1>:<n2>i<p>/} >> : retourne la liste des nombres compris entre <n1> et <n2>, avec un pas <p>. 
\end{itemize}
Trois options sont \'egalement propos\'ees :
\begin{itemize}
 \item[-] << \textbf{-e <e>} >> : exclut la liste des nombres se trouvant dans l'ensemble <e>,
 \item[-] << \textbf{-n <s>} >> : chaque nombre du flux d'entr\'ee est s\'epar\'e par le caract\`ere <s>. Par d\'efaut, le s\'eparateur est l'espace.
 \item[-] << \textbf{-N} >> : chaque nombre du flux d'entr\'ee est s\'epar\'e par le caract\`ere de retour \`a la ligne.
\end{itemize}
\end{itemize}

\section{Utilit\'e de ces programmes}

Les dix programmes pr\'esent\'es ci-dessus sont relativement simples \`a utiliser. Chacun ne pr\'esente qu'une fonction principale ainsi que quelques options.
Ce type de programme est donc appr\'eci\'e des programmeurs, par sa facilt\'e et sa rapidit\'e d'ex\'ecution. Dans le chapitre \ref{chap:tests et optimisations}
nous pr\'esenterons en d\'etail les performances obtenues.
\newline
Exemples d'utilisation de ces programmes :
\begin{itemize}
 \item[-] \'Ecrire dans result.txt l'ensemble des multiples de 12 contenus data.txt :
 \begin{verbatim} numgrep /m12/ data.txt >result.txt \end{verbatim}
 \item[-] Normaliser (entre 0 et 1) les nombres contenus dans data.txt:
 \begin{verbatim} numnormalize data.txt \end{verbatim}
 \item[-] G\'en\'erer un ensemble d'\'echantillons dans l'intervalle [1,100] avec un pas de 0.2 :
 \begin{verbatim} numrange /1:100:0.2/ \end{verbatim}
 \item[-] En choisir un al\'eatoirement :
 \begin{verbatim} numrandom /1:100:0.2/ \end{verbatim}
\end{itemize}

Ces exemples ne sont que quelques cas d'utilisations des fonctions de la biblioth`eque num-utils-ng que nous avons optimis\'e.


