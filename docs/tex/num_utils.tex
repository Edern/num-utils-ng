\chapter{La biblioth\`eque num-utils}
\label{chap:La bibliotheque num-utils}

\section{Pr\'esentation}

La biblioth\`eque de calcul num\'erique qu'il nous fallait am\'eliorer s'appelle num-utils. Elle est \'ecrite dans un langage
interpr\'et\'e, le Perl.
Elle impl\'emente un certain nombre d'algorithmes num\'eriques travaillant sur des flots de donn\'ees.
Ces algorithmes sont r\'epartis dans dix programmes diff\'erents s'intitulant : numaverage, numbound, numgrep, numinterval, numnormalize,
numprocess, numrandom, numrange, numround et numsum.

La premi\`ere am\'elioration que l'on pouvait apporter \`a la biblioth\`eque \'etait de passer d'un langage interpr\'et\'e en un langage compil\'e.
Nous avons ainsi cod\'e, en langage C, l'ensemble de ces utilitaires.


\section{Fonctionnalit\'es propos\'ees par ces dix programmes}

Cette partie explique en d\'etail les fonctionnalit\'es propos\'es par ces diff\'erents programmes.
Les dix programmes constituant la biblioth\`eque num-utils-ng s'ex\'ecutent en ligne de commande dans un terminal Unix (shell).
Pour chacun de ces programmes, except\'e numrandom qui ne prend aucune donn\'ee en entr\'ee, les donn\'ees sont lues \`a partir 
de l'entr\'ee standard (stdin) ou \`a partir d'un fichier.
\newline
\begin{itemize}
 \item[\textbullet] \textbf{numaverage :} numaverage retourne la moyenne des nombres relatifs pass\'es en entr\'ee.
Ce programme propose cinq options :
\begin{itemize}
  \item << \textbf{-i} >> : retourne la partie enti\`ere de la moyenne.
  \item << \textbf{-I} >> : retourne la partie d\'ecimale de la moyenne.
  \item << \textbf{-m} >> : retourne le nombre qui appara\^it le plus souvent.
  \item << \textbf{-M} >> : retourne le nombre m\'edian.
  \item << \textbf{-l} >> : retourne le nombre m\'edian le plus petit (utile quand le nombre total d\'el\'ements est pair).
\newline
\end{itemize}

 \item[\textbullet] \textbf{numbound :} numbound renvoie le plus grand nombre pass\'e en entr\'ee.
Ce programme propose une seule option :
\begin{itemize}
  \item << \textbf{-l} >> : retourne le plus petit nombre pass\'e en entr\'ee.
\newline
\end{itemize}

 \item[\textbullet] \textbf{numgrep :} 
 \newline
 \item[\textbullet] \textbf{numinterval :} numinterval calcule et affiche l'intervalle entre un nombre et le suivant du flux d'entr\'ee. 
 \newline
 \item[\textbullet] \textbf{numnormalize :} numnormalize prend un ensemble de nombres en entr\'ee et retourne l'ensemble de ces nombres, 
normalis\'es entre 0 et 1 par d\'efaut.
 \newline
 \item[\textbullet] \textbf{numprocess :} numprocess effectue une liste d'op\'erations sur l'ensemble des nombres pass\'es en entr\'ee.
Les op\'erations possibles sont :
\newline
\newline
\begin{tabular}{|l|c|r|}
  \hline
  Symbole &  Op\'eration\\
  \hline
  << \textbf{+} >> & Addition \\
  << \textbf{-} >> & Soustraction \\
  << \textbf{*} >> & Multiplication \\
  << \textbf{\%} >> & Division \\
  << \textbf{\^} >> & Puissance \\
  << \textbf{sqrt} >> & Racine carr\'ee \\
  << \textbf{sin} >> & Sinus \\
  << \textbf{cos} >> & Cosinus \\
  \hline
\end{tabular}
\newline
\newline
 \item[\textbullet] \textbf{numrandom :}
 \newline
 \item[\textbullet] \textbf{numrange :}
 \newline
 \item[\textbullet] \textbf{numround :}
 \newline
 \item[\textbullet] \textbf{numsum :}
 \newline
\end{itemize}

Ces dix programmes proposent chacun une option << \textbf{-h} >> permettant \`a l'utilisateur d'obtenir une aide sur le fonctionnement du 
programme sous la forme d'une page de manuel (man pages en anglais) .

\section{Utilit\'e de ces programmes}

